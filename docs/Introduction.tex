\chapter{Introdução}\footnote{Este relatório foi produzido com recurso ao template \LaTeX\ para teses e dissertações da Universidade da Beira Interior \cite{template-ubi-latex}, uma versão não oficial mantida pela comunidade.} \label{chap:intro}

\section{História e Enquadramento} \label{sec:intro_historia}
O jogador encontra-se num labirinto, com uma lanterna que ilumina o caminho e uma citação pouco esclarecedora. Percebe que a mesma pessoa que vai sair dali, não é a mesma que entrou.

\section{Motivação} \label{sec:intro_motivacao}
A escolha de desenvolver um jogo de labirinto 3D foi motivada por diversos fatores:

\begin{itemize}
    \item \textbf{Geração Procedural}: Implementar algoritmos de geração procedural, nomeadamente o algoritmo de Kruskal, permite criar labirintos únicos e imprevisíveis em cada execução.
    \item \textbf{Aplicação de Computação Gráfica}: O projeto requer a aplicação prática de transformações 3D, sistemas de iluminação, mapeamento de texturas, e programação de shaders GLSL.
    \item \textbf{Interatividade}: A implementação de controlos de câmara em primeira pessoa e deteção de colisões proporciona uma experiência imersiva.
    \item \textbf{Networking}: A inclusão de funcionalidades de rede (modo host/cliente) adiciona complexidade técnica e demonstra conhecimentos de programação de sistemas.
    \item \textbf{Desafios Técnicos}: O projeto apresenta desafios interessantes em áreas como otimização de renderização, gestão de recursos, e arquitetura de software.
\end{itemize}

\section{Objetivos} \label{sec:intro_objetivos}
Os principais objetivos deste projeto são:

\begin{itemize}
    \item Desenvolver um motor de renderização 3D funcional utilizando OpenGL.
    \item Implementar geração procedural de labirintos através do algoritmo de Kruskal.
    \item Criar um sistema de iluminação dinâmica com efeito de lanterna (\textit{flashlight/spotlight}).
    \item Implementar comunicação em rede entre dois jogadores (modo host e cliente).
    \item Desenvolver sistema de camera em primeira pessoa com controlos intuitivos.
    \item Adicionar elementos visuais como ambiente exterior, árvores e portal de saída.
    \item Implementar renderização de texto para interface com o utilizador.
    \item Aplicar texturas e materiais realistas aos elementos do jogo.
\end{itemize}

\section{Estrutura do Relatório} \label{sec:intro_estrutura}
Este relatório está organizado da seguinte forma:

\begin{itemize}
    \item O \textbf{Capítulo \ref{chap:tecnologias}} descreve as tecnologias e bibliotecas utilizadas no desenvolvimento do projeto.
    \item O \textbf{Capítulo \ref{chap:implementacao}} detalha o desenvolvimento e implementação, incluindo a gestão do projeto, descrição das classes e funções principais.
    \item O \textbf{Capítulo \ref{chap:conclusao}} apresenta as conclusões, trabalho realizado, limitações e sugestões para trabalho futuro.
\end{itemize}
