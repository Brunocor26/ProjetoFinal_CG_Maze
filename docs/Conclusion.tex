\chapter{Conclusão e Trabalho Futuro} \label{chap:conclusao}

\section{Conclusão} \label{sec:conc_conclusao}

Este projeto atingiu com sucesso os seus objetivos principais, resultando numa aplicação 3D interativa funcional que demonstra a aplicação prática de conceitos fundamentais de computação gráfica.

\subsection{O que se Fez}
As seguintes funcionalidades foram implementadas com sucesso:

\begin{itemize}
    \item \textbf{Motor de Renderização 3D}: Sistema completo baseado em OpenGL com \textit{pipeline} programável e shaders GLSL customizados.
    
    \item \textbf{Geração Procedural}: Implementação do algoritmo de Kruskal para geração de labirintos perfeitos, permitindo experiências únicas em cada execução.
    
    \item \textbf{Sistema de Iluminação}: Iluminação dinâmica tipo lanterna com atenuação realista e efeito de névoa atmosférica, criando uma atmosfera imersiva.
    
    \item \textbf{Networking Multiplayer}: Sistema de comunicação TCP entre host e cliente com mecânica de desbloqueio progressivo.
    
    \item \textbf{Câmara FPS}: Controlos de primeira pessoa intuitivos com movimento WASD e rotação com rato.
    
    \item \textbf{Ambiente 3D Completo}: Para além do labirinto, inclusão de ambiente exterior com árvores procedurais, portal arquitetônico e texturas realistas.
    
    \item \textbf{Interface de Utilizador}: Sistema de renderização de texto para diálogos e feedback ao jogador.
    
    \item \textbf{Física Básica}: Deteção de colisões e impedimento de atravessamento de paredes.
    
    \item \textbf{Texturas PBR}: Utilização de texturas de alta qualidade com mapas de normal e rugosidade para realismo visual.
    
    \item \textbf{Transição de Cor Progressiva}: Sistema de mudança gradual da cor do ambiente baseado na proximidade ao portal, proporcionando feedback visual atmosférico ao jogador.
\end{itemize}

O desenvolvimento deste projeto proporcionou aprendizagem valiosa em áreas como:
\begin{itemize}
    \item Programação gráfica com OpenGL moderno
    \item Arquitetura de software para aplicações em tempo real
    \item Álgebra linear aplicada (transformações 3D, projeções)
    \item Programação de shaders GLSL
    \item Gestão de recursos (texturas, \textit{meshes}, shaders)
    \item Networking com sockets TCP
    \item Algoritmos de geração procedural
\end{itemize}

\section{Limitações e Trabalho Não Realizado} \label{sec:conc_limitacoes}

Apesar do sucesso geral do projeto, algumas limitações foram identificadas:

\begin{itemize}
    \item \textbf{Documentação Doxygen Incompleta}: Apenas algumas classes possuem documentação Doxygen completa.
    
    \item \textbf{Otimização}: Não foram implementadas técnicas avançadas de otimização como \textit{frustum culling} ou \textit{occlusion culling}.
    
    \item \textbf{Sincronização de Rede}: O networking é assíncrono básico sem sincronização completa de posições entre jogadores.
    
    \item \textbf{Áudio}: Não foi implementado sistema de som ou música.
    
    \item \textbf{Modelos Complexos}: Os objetos 3D (árvores, portal) são procedurais simples, não modelos importados detalhados.
    
    \item \textbf{Interface Gráfica}: A interface é maioritariamente textual, sem HUD gráfico elaborado.
\end{itemize}

\section{Trabalho Futuro} \label{sec:conc_futuro}

O projeto tem margem significativa para evolução. Sugestões para versões futuras incluem:

\begin{itemize}
    \item \textbf{Níveis Múltiplos}: Sistema de progressão com múltiplos labirintos de dificuldade crescente.
    
    \item \textbf{Algoritmos Alternativos}: Implementação de outros algoritmos de geração (Prim, Recursive Backtracking, Eller) com seleção pelo utilizador.
    
    \item \textbf{Sincronização Completa de Rede}: Visualização da posição do outro jogador em tempo real na cena 3D.
    
    \item \textbf{Sistema de Áudio}: Música de fundo, efeitos sonoros de passos, sons ambientes.
    
    \item \textbf{Power-ups e Obstáculos}: Elementos de \textit{gameplay} adicionais como chaves, portas trancadas, armadilhas.
    
    \item \textbf{Minimapa}: Interface 2D mostrando o labirinto e posição do jogador.
    
    \item \textbf{Modelos 3D Importados}: Substituir geometria procedural por modelos artísticos de qualidade.
    
    \item \textbf{Shaders Avançados}: Implementar \textit{shadow mapping}, \textit{ambient occlusion}, \textit{bloom}, e outros efeitos visuais.
    
    \item \textbf{Configuração Dinâmica}: Menu de configurações para ajustar resolução, qualidade gráfica, tamanho de labirinto.
    
    \item \textbf{Suporte VR}: Adaptação para \textit{Virtual Reality} para experiência ainda mais imersiva.
    
    \item \textbf{Sistema de Partículas}: Efeitos visuais como poeiras, fagulhas, ou partículas mágicas no portal.
    
    \item \textbf{IA de Inimigos}: Adicionar entidades controladas por IA que perseguem o jogador.
\end{itemize}

\section{Considerações Finais} \label{sec:conc_finais}

Este projeto demonstrou com sucesso a aplicação de conceitos de computação gráfica numa aplicação prática e interativa. A combinação de geração procedural, renderização 3D, iluminação dinâmica e networking resultou numa experiência de jogo funcional e tecnicamente interessante.

O desenvolvimento reforçou a importância de uma arquitetura de software bem estruturada, da gestão cuidadosa de recursos gráficos, e da compreensão profunda dos conceitos matemáticos subjacentes à computação gráfica 3D.
